\documentclass[a4paper,11pt,notumble]{leaflet}
%delete tumble for reverse printing on second side
\usepackage[utf8]{inputenc}
\usepackage[ngerman]{babel}
\usepackage{ae}
\usepackage{amsmath}
\usepackage{setspace}
\usepackage{graphicx}

\graphicspath{{gfx/}}

%%%  #### end preamble ###
%%%%%%%%%%%%%%%%%%%%%%%%%%%%
%leaflet-stuff
%\renewcommand*\foldmarkrule{.3mm}
%\renewcommand*\foldmarklength{5mm}
%\CutLine{6}%  Dotted line with scissors
%\CutLine*{1}%Dotted line without scissors

\hyphenation{Nutzer mit-ge-brach-ten funk-tio-nier-en-de}
\begin{document} 
% some definitions a la Bruno
\newenvironment{pandora}{\fontfamily{pnb10}\selectfont}{}
\newenvironment{panfett}{\fontfamily{pnss10}\selectfont}{}
\newenvironment{ebmr}{\fontfamily{ebmr10}\selectfont}{}
\newenvironment{anb}{\fontfamily{anb10u}\selectfont}{}
\setlength{\fboxsep}{0.2cm}  % luft zum Rand
\setlength{\parindent}{0pt}  % kein Erstzeileneinzug

\newcommand{\VT}[3]{{\tt #1} \textsl{#2} {\bf #3}\\[2mm]}


%\vspace*{5mm}

\begin{center}
\Huge{\sc Einladung}\\[3mm]
{\small zur}\\[1mm]
\large{\sc Troisdorfer Linux User Group}\\[1mm]
{\small - ein Club vom}\\[1mm]
\large{\sc Ortsverein AWO-Troisdorf-Mitte} \\[1mm] 
  \includegraphics[width=2cm]{awo-logo}\\[5mm]
{\large Wir treffen uns jeden} \\[0mm]

  \textbf{\Huge{1. Donnerstag\\[2mm] }} %
  im Monat, der nicht Feiertag ist, im\\[5mm]
% \includegraphics[height=3.0cm]{Swfrddyclip}\\[5mm]
 \textbf{\sc Agnes-Klein-Zentrum\\[2mm]
AWO-Toisdorf-Mitte\\
Wilhelm-Hamacher-Platz 12\\
53840 Troisdorf\\[5mm]}
 \includegraphics[width=3cm]{trolug-logo}\\[4mm]
\textbf{\large{Beginn ist pünktlich um\\[1mm]}}
\textbf{\Huge{19:00 Uhr}} \\[1mm]
\textbf{\large{Ende um 21:00 Uhr}}\\[2mm]
\textsf{\large http://www.trolug.de/ }\\[2mm]
\end{center}

\vfill
\subsection{Wer ist TroLUG?}
Die {\sc Troisdorfer Linux User Group} (kurz TroLUG) wurde von Jonas Stein im Januar 2009 als Stammtisch gegründet. 
Das Stadtbrauhaus eignete sich nur bedingt für Vorträge. Frank Böhm stellte den Kontakt zur AWO her und die TroLUG wurden nach einiger Zeit Club der AWO.\\
Jetzt steht der TroLUG ein Konferenzraum mit moderner Technik zur Verfügung, in dem sich 15-20 Frauen und Männer zwischen 20 und 70 Jahren treffen.



\subsection{Was bietet die TroLUG?}
\begin{itemize}
\item Vorträge zu freier Software und speziell Linux
\item gegenseitige Hilfe bei ganz alltäglichen Problemen
\item Vorstellung von Fachliteratur und Fachzeitschriften
\item vorbildliche technische Infrastruktur
\item eine Umgebung, in der sich die interessierten Frauen und Männer aller Altersklassen wohlfühlen
\item Mailingliste, Newsgroup, aktuelle Webseite
\item elektronische Signaturen gpg und CAcert
\end{itemize}



\subsection{Welche Distribution empfiehlt die TroLUG?}
Es gibt viele spezielle Zusammenstellungen von Linuxprogrammen zu einer Distribution. Damit wir bei Problemen schneller helfen können, einigten wir uns auf (K)Ubuntu. Kubuntu und Ubuntu unterscheiden sich nur in der Bildschirmdarstellung im ersten Moment nach der Installation. Ihr Vorteil ist, dass die Hardwareerkennung sehr gut funktioniert und man im Internet als Einsteiger sehr leicht Hilfe bekommt.

Wer es sich zutraut kann natürlich beliebige andere Distributionen verwenden. 

\subsection{Wie kann man sich einbringen?}
\begin{itemize}
\item Es gibt viele kleine, aber wichtige Aufgaben, die oft nicht einmal etwas mit dem PC zu tun haben. Der Moderator kann Ihnen sagen, wo gerade dringend Hilfe gebraucht wird.
\item Wir freuen uns über neue Vorträge. Gibt es ein Wunschthema? Kennen Sie sich mit etwas gut aus? Trauen Sie sich und stellen Sie es der Gruppe vor! 
\item Alle Helfer arbeiten ehrenamtlich, aber Netzwerktechnik, Reinigung, Sanitäranlagen, Beamer, Flipcharts, Papier, Stifte und vieles mehr muss aus Spenden bezahlt werden. \emph{Bitte geben Sie am Ende eines Vortrags der AWO eine Spende}, damit wir gut arbeiten können. 
\item Sie könnten AWO-Mitglied werden und schon mit niedrigen Beiträgen vielen Projekten sehr helfen. 
\end{itemize}


\newpage
\section{Sponsoren:}

\includegraphics[width=\textwidth]{SponsorenCollage}
%\vfill

\includegraphics[height=8cm]{trolugtux1.pdf}\\


\newpage
\section{Termine}
Jeden Monat gibt es einen Vortrag oder Workshop.
{\flushleft{\small

\VT{2010-02-04}{Dipl.-Ing. Ingo Wichmann (Linuxhotel)} {vim - einmal 60 Minuten lernen, täglich Zeit sparen}
\VT{2010-03-04}{Maic Striepe (GWG Troisdorf)} {Scribus}
\VT{2010-04-01}{Dr. Uwe Ziegenhagen}  {Einführung in das LaTeX Textsatzsystem}
\VT{2010-05-06}{Dimitri Asarowski (EDV-Lotse)}{Workshop: Netzwerktools unter Linux}
\VT{2010-06-03}{verschiedene}{Openstreetmap-Kurzvortragsabend}
Sondertermin:\\
\VT{2010-06-26 Sa}{verschiedene}{Kartographieren (Mapping Party)}
\VT{2010-08-21 bis 2010-08-22}{Messestand der TroLUG}{FrOSCon Sankt Augustin}




}}

Auf unserer Webseite und in der Mailingliste wird stets über aktuelle und künftige Vorträge informiert. 
Zu vielen Vorträgen gibt es Links, Quelltexte oder Bildschirmfotos, um zuhause alles nochmal in Ruhe nachzuarbeiten.

Unser Programm soll eine Mischung aus Themen sein, die absolute Einsteiger ansprechen, aber auch für Fortgeschrittene manchmal neu sind.

\vfill

\fbox{\parbox{7.5cm}{\setlength{\parindent}{0pt} \textsf{V.i.S.d.P.
      Jonas Stein\\[0.5mm] 
      Canisiusstraße 9, 53840 Troisdorf\\} %
    \vspace*{-2.5mm} %
    \scriptsize{\flushleft{Satz und technische Realisierung:    Jonas Stein
        Dr. M. H. Fröhlich und B. Hopp mit \LaTeXe }}}}

\newpage
\section{Notizen:}
\hline

%% this is the end
\end{document}  
%%  die (noch) aufrecht (=landscape) Seiten müssen noch 
%% _rotiert_ werden damit die jeweils drei Spalten auf
%% eine A4-seite sortiert werden! Beim doppelseitigen Ausdruck "Kurze
%% Seite" zum Umblättern wählen.
%% 
%%  Local Variables: 
%%  mode: latex
%%  TeX-master: t
%%  End: 
